% !TEX root = ../Thesis.tex

\chapter{Introduzione}
Negli ultimi anni, Rust è emerso come uno dei linguaggi di programmazione più promettenti e apprezzati dalla comunità degli sviluppatori, guadagnando popolarità in ambiti che richiedono elevate prestazioni e garanzie di sicurezza della memoria. Creato da Mozilla Research e successivamente adottato da grandi aziende come Microsoft, Google e Amazon, Rust offre un modello di gestione della memoria innovativo che elimina la necessità di un Garbage Collector, garantendo l'assenza di errori comuni come buffer overflow, dangling pointer e data race. Queste caratteristiche lo rendono particolarmente adatto per lo sviluppo di sistemi critici come sistemi operativi e infrastrutture di rete.

Tuttavia, Rust si discosta notevolmente dai linguaggi object-oriented tradizionali come Java, che da decenni domina lo sviluppo applicativo. Java, con la sua gestione automatica della memoria tramite Garbage Collector e il suo solido supporto al polimorfismo tramite ereditarietà e interfacce, ha semplificato lo sviluppo software per generazioni di programmatori. Il confronto tra questi due linguaggi non è solo accademico, ma ha implicazioni pratiche per sviluppatori che devono valutare quale tecnologia adottare.

Questa tesi si propone di analizzare sistematicamente le differenze fondamentali tra Rust e Java, concentrandosi su due aspetti chiave: la gestione della memoria e il supporto al polimorfismo. Attraverso un'analisi comparativa dei due aspetti, il lavoro mira a evidenziare le diverse scelte progettuali alla base di ciascun linguaggio e le loro implicazioni pratiche. 

Il metodo di analisi si basa sull'esame diretto delle caratteristiche linguistiche, supportato da esempi di codice comparativi che illustrano come concetti simili siano implementati in modo diverso nei due linguaggi. L'attenzione è rivolta non solo alle differenze sintattiche, ma anche alle implicazioni che ciascun approccio comporta sullo sviluppo del software.

La tesi è strutturata come segue: 
\begin{itemize}
    \item Capitolo 2: Panoramica dei tipi di dati nei due linguaggi, seguita dalla presentazione dei concetti fondamentali, come \texttt{struct} ed \texttt{enum} in Rust, necessari per la corretta comprensione dei capitoli successivi. 
    \item Capitolo 3: Analisi delle differenze nella gestione della memoria tra Rust e Java, con particolare attenzione ai concetti di ownership e borrowing in Rust, confrontati con il modello di Garbage Collection di Java.
    \item Capitolo 4: Analisi delle differenze nel supporto al polimorfismo, esaminando i meccanismi statici e dinamici offerti da entrambi i linguaggi. Il capitolo si conclude con una sezione che mostra, mediante un esempio pratico, come sia possibile estendere il comportamento nei due linguaggi sfruttando i rispettivi meccanismi di polimorfismo.
    \item Capitolo 5: In quest'ultimo capitolo vengono riassunti i principali risultati dell'analisi comparativa, evidenziando i punti di forza e le limitazioni di ciascun linguaggio nei due ambiti considerati. 
\end{itemize}
